\documentclass[pdftex,12pt,a4paper]{article}

\usepackage{graphicx}  
\usepackage[margin=2.5cm]{geometry}
\usepackage{breakcites}
\usepackage{indentfirst}
\usepackage{pgfgantt}
\usepackage{pdflscape}
\usepackage{float}
\usepackage{epsfig}
\usepackage{epstopdf}
\usepackage[cmex10]{amsmath}
\usepackage{stfloats}
\usepackage{multirow}

\renewcommand{\refname}{REFERENCES}
\linespread{1.3}

\usepackage{mathtools}
%\newcommand{\HRule}{\rule{\linewidth}{0.5mm}}
\thispagestyle{empty}
\begin{document}
\begin{titlepage}
\begin{center}
\textbf{}\\
\textbf{\Large{ISTANBUL TECHNICAL UNIVERSITY}}\\
\vspace{0.5cm}
\textbf{\Large{COMPUTER ENGINEERING DEPARTMENT}}\\
\vspace{2cm}
\textbf{\Large{BLG 222E\\ Computer Organization \\ Project 1}}\\
\vspace{2.8cm}
\begin{table}[ht]
\centering
\Large{
\begin{tabular}{lcl}
\textbf{PROJECT DATE}  & : & 24.05.2023\\
\end{tabular}}
\end{table}
\vspace{1cm}
\textbf{\Large{GROUP MEMBERS:}}\\
\begin{table}[ht]
\centering
\Large{
\begin{tabular}{rcl}
150220762  & : & Muhammed Yusuf Mermer (Group Representative)  \\
150210071  & : & Emre Çamlıca \\
150200091  & : & Hakan Duran \\
\end{tabular}}
\end{table}
\vspace{2.8cm}
\textbf{\Large{SPRING 2023}}

\end{center}

\end{titlepage}

\thispagestyle{empty}
\addtocontents{toc}{\contentsline {section}{\numberline {}FRONT COVER}{}}
\addtocontents{toc}{\contentsline {section}{\numberline {}CONTENTS}{}}
\setcounter{tocdepth}{4}
\tableofcontents
\clearpage

\setcounter{page}{1}
\section{INTRODUCTION}
In this project our purpose was to design a hardwired control unit. To achive 
this we firstly implemented small components of this system. Yusuf did the fetch and decode parts, Hakan did the instructions without memory reference and Emre did the instructions with memory reference. Yusuf also wrote the test bench and tested the system with Hakan.

At first, Yusuf has designed the fetch cycle. Then Hakan and Emre designed the execute part. Hakan and Yusuf then tested the project and corrected the errors.


\section{IMPLEMENTATIONS AND EXPLANATIONS }
\subsection{Fetch Cycle}

\subsection{Instructions With Address Reference}

\subsection{Instructions Without Address Reference}
    Instructions without address references have four seperate fields whose functionalities differ. Each one has 4 bits. Opcode determines which operation
will be carrying out. DSTREG means destination register, SREG1 and SREG2 are source registers. There are 10 instruction types.
\subsubsection{AND}
    AND operation's opcode is 0x00. It carries out by determining ALU_Funsel as 0111.
\subsubsection{OR}
    OR operation's opcode is 0x01. It carries out by determining ALU_Funsel as 1000.
\subsubsection{NOT}
    NOT operation's opcode is 0x02. It carries out by determining ALU_Funsel as 0010.
\subsubsection{ADD}
    ADD operation's opcode is 0x03. It carries out by determining ALU_Funsel as 0100.
\subsubsection{SUB}
    SUB operation's opcode is 0x04, it is responsible for substract operations. It carries out by determining ALU_Funsel as 0101.
\subsubsection{LSR}
    LSR operation's opcode is 0x05, it is responsible for logical shift right operations. It carries out by determining ALU_Funsel as 1100.
\subsubsection{LSL}
    LSL operation's opcode is 0x06, it is responsible for logical shift left operations. It carries out by determining ALU_Funsel as 1011.
\subsubsection{INC}
    INC operation's opcode is 0x07, it is responsible for increment operations. It carries out by determining ALU_Funsel as 0100.
It takes 3 cycles because in order to increment and then send to ALU, we should first create a register
which has value 1. Making it takes 2 cycle, first cycle for using a temporary register which has value 0.
Then in second cycle, we make it 0001 with incrementing and use it in ALU.
\subsubsection{DEC}
    DEC operation's opcode is 0x08, it is responsible for decrement operations. It carries out by determining ALU_Funsel as 0101.
It takes 3 cycles because in order to decrement and then send to ALU, we should first create a register
which has value 1. Making it takes 2 cycle, first cycle for using a temporary register which has value 0.
Then in second cycle, we make it 0001 with incrementing and use it in ALU.
\subsubsection{MOV}
    MOV operation's opcode is 0x0B. It carries out by determining ALU_Funsel as 0000.


\section{OVERALL DESIGN PHOTO}








\section{SIMULATION RESULTS}




\section{DISCUSSION}
    In this project, we have implemented an ALU system which can take instructions defined in
Instruction Format. In order to achieve that, we have implemented a counter and by using counter's 
timing signal output, we have implemented fetch-decode operations and instructions' actual operations on
registers in register file and address register file. 

    We think we greatly achieve what we wanted. We successfully implemented fetch-decode operations
and instructions' actual operations defined in their OPCODE part. We've loaded our memory to instruction
register and used it in ALUSystem's components. While our code has worked, we couldn't run the last ST
operation successfully. While we achieve the adding numbers in memory from 0xB0 to 0xBA, we couldn't store it
in memory. It is because we couldn't find a way to write value into the memory. But our success can be seen
from the simulation bar.

    While doing this homework, we have encountered with a lot of difficulties. One of them is our one teamfriend, Emre,
were in another city, it maked our job difficult, but we've overcomed it by giving less tasks to our teamfriend
that is in another city.

    Another big difficult was debugging. Finding the root of problems has caused us to lose a lot of days.
We've encountered with a lot of different problems. Syncing the clock with code was the hardest task.
Another big problem was ensuring the security. Without security, we have overwrite some registers and
didn't realize it. Enhancing security has provideded our code to be safe. 

    The other issue was the limited time. 5 ns clock was not enough for us to simulate all the loop. In order to
overcome it, we have adjusted the simulation time in setting by changing xsim.simulation.runtime to 4000 ns. Default is
1000 ns.

    We have learned how an instructor is performed in the computer, and its relates with clock cycles and ALU components.


\section{CONCLUSION}
In this project we designed a hardwired control unit with the help of the ones we created in the first project. 

\end{document}